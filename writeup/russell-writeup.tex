% Russell Writeup

% understanding algorithms - complexity of quicksort
The complexity of Quicksort can vary considerably depending on the layout of the
data being sorted and on how the pivot value is picked. Carefully picking a
pivot value can be considered optimization and in our algorithms we didn't do
anything special to pick ours. This is a divide and conquer algorithm, and on
average can do O(n logn). The conquer part is a platform-specific complexity
because sometimes concatenating lists or arrays together can be constant, but is
usually a linear operation. The divide happens around a pivot value, and if the
pivot value is greater than half of the array and less than the other half there
will be a logarithmic complexity for the divide. If the pivot value causes one
side of the division to be a lot larger than the other, the complexity moves
toward quadratic. The nice thing about this algorithm, though, is it doesn't
need to do as many comparisons as Merge Sort. Once it has done its full
recursion it is simply concatenating results together. This may be the primary
reason for Quicksort's victory.
