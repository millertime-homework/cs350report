%Copyright 2011 Russell Miller & Ben Carr
% Please see included LICENSE file

\documentclass[12pt]{amsart}
\usepackage{geometry} % see geometry.pdf on how to lay out the page. There's lots.
\usepackage{amsmath}
\usepackage{ amssymb }
\geometry{a4paper} % or letter or a5paper or ... etc

\title{CS350 Term Project (Proposal)}
\author{Ben Carr and Russell Miller}
\date{\today} % delete this line to display the current date

%%% BEGIN DOCUMENT
\begin{document}

\maketitle


\section*{}
\subsection*{}


Our term project will focus on sorting algorithms. We would like to compare several sorting algorithms:
\begin{itemize}
\item Bubble, average  $O(n^2)$, worst-case $O(n^2)$\\
\item Insertion, average $O(n^2)$, worst-case $O(n^2)$\\
\item Quick, average $O(n lg n)$ worst-case $O(n^2)$\\
\item Merge, average and worst-case $O(n lg n)$\\
\end{itemize}
(We may also try bucket sort (average $O(n)$, worst-case $O(n^2)$), time permitting, as it is more complex to implement.)\\ \\
Our goal is to implement each of these algorithms in Python, Java, C++ and Haskell, (approximately 20 algorithms in all). Russell will write the Python and Java code, and Ben will write C++ and Haskell. We will test our algorithms on unordered lists of integers. Each algorithm will sort the same 1000 lists of size \\
$n = \{10 \; 50 \; 100 \;500\; 1,000\; 5,000\; 10,000\; 50,000 \;100,000\; 500,000 \\1,000,000\; 5,000,000 \;1,000,000,000\}$ that will be read from an external data file. The data we collect from these tests will be the execution time taken for the algorithm to sort the list. Once we have collected this data, several comparisons can be made. I would like to compare the following:
\begin{itemize}
\item How algorithm $A$ in programming language $L$ behaves as the size $n$ of the list gets larger. \\
\item How algorithm $A_{n}$ compares to $A_{n}$ in languages $L_{n}, L_{m}$. (Same algorithm, different language)\\
\item The consistency or average of the algorithms after executing on list of size $n$. \\
\end{itemize}
We are creating a generator program that will aide in this process by:\\
\begin{itemize}
\item creating the sample data files (lists of n integers)
\item executing and timing our algorithms 
\item writing the results to a .csv file\\
\end{itemize}
Our final report will have a discussion of our algorithms, and comparisons of their performance. I would like to graph the comparisons of the algorithms against each other, and also create "consistency graphs" that show that the expected complexities are consistent. We will also discuss how complexity theory shows up in practice.\\ \\
Our goals for the next three weeks are as follows:\\ \\
WEEK 1:\\ \\
1. make a generator that:
\begin{itemize}
\item creates sample data in an external file
\item executes and times our algorithms (in any language)
\item writes the results to a .csv file
\end{itemize}
2. Finish at least two algorithms in C++ and Python that we can use to test the generator. \\
3. Test .csv file in spreadsheet program\\ \\
WEEK 2:\\ \\
1. Finish the rest of the C++ and Python algorithms\\ 
2. Graph results of all implemented algorithms\\
3. Start Java and Haskell implementation\\ \\
WEEK 3:\\ \\
1. Finish writing and graphing Java and Haskell algorithms\\
2. Analyze/Discuss our findings\\
3. Write up our results for term paper.\\
\\
So far we have been able to start the generator program that will run the executables, and also
have the random list-generator working.\\ \\
If you'd like to see our current progress, 
please visit our git repository: 
\begin{verbatim}
https://github.com/millertime-homework/cs350report
\end{verbatim}
\end{document}
